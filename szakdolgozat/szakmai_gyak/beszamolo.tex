\documentclass[12pt,a4paper]{article}

\usepackage{styles/dolgozat}

\usepackage{listings}
% \usepackage{styles/cpp}
\usepackage{styles/python}

\usepackage{hyperref}

\begin{document}

\include{cover/cimlap}

\newpage

\pagestyle{empty}

\cleardoublepage
\pagenumbering{gobble}
\tableofcontents
\cleardoublepage
\pagenumbering{arabic}

\newpage

\pagestyle{fancy}

\section{A feladat bemutatása}

Le kellene írni röviden, hogy mi volt a cél a játékkal.

\section{Technológia kiválasztása}

Ebben érdemes részletezni a Godot Game Engine-t, és tenni hozzá hivatkozásokat is, mint például \cite{salmela2022game,holfeld2023relevance}.

\section{A tervezés folyamata}

A részfeladatokat úgy általában mindenkinek a saját szemszögéből kellene bemutatnia, kihangsúlyozva a saját részt.

\subsection{Előkészületek}

Ezt csak úgy példának írtam, hogy lehet szépen strukturálni a dokumentumot.

\subsubsection{Telepítés}

Szintén csak példa. Egyáltalán nem kell ragaszkodni a mintában szereplő címekhez, alcímekhez.

\section{Implementáció}

Jó, hogy ha lesz a dokumentumban ábra, ami a szövegből hivatkozásra is kerül, mint például \aref{fig:cimer}.

\begin{figure}[h]
	\centering
	\includegraphics[scale=0.1]{images/me_logo.png}
	\caption{A Miskolci Egyetem címere.}
	\label{fig:cimer}
\end{figure}

Ahol indokolt, ott táblázatok is jó, hogy ha vannak (például \ref{tab:minta}. táblázat).

\begin{table}[h]
	\centering
	\caption{Minta táblázat. A táblázat felirata a táblázat felett kell legyen!}
	\label{tab:minta}
	\begin{tabular}{l|c|c|}
		a & b & c \\
		\hline
		1 & 2 & 3 \\
		4 & 5 & 6 \\
		\hline
	\end{tabular}
\end{table}

Mivel a Godot Script-nek Python-szerű a szintaxisa, ezért lehet Python kódkörnyezetet használni hozzá, például:

\begin{python}
func add_item(name: String, amount: int = 1):
    if not items.has(name):
        items[name] = 0
        items[name] += amount

func remove_item(name: String, amount: int = 1):
    if items.has(name):
        items[name] = max(0, items[name] - amount)
        if items[name] == 0:
            items.erase(name)
\end{python}

\section{Tesztelés}

\section{Összegzés}

Ide érdemes leírni a munkálatokkal kapcsolatos tapasztalatokat.

\clearpage

\addcontentsline{toc}{section}{Hivatkozások}
\bibliographystyle{plain}
\bibliography{beszamolo.bib}

\newpage

\include{cover/hasznalati}

\end{document}
