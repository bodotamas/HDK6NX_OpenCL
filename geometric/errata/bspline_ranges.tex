\documentclass[a4paper,12pt]{article}

% Set margins
\usepackage[hmargin=2.5cm, vmargin=2cm]{geometry}

\frenchspacing

% Language packages
\usepackage[utf8]{inputenc}
\usepackage[T1]{fontenc}
\usepackage[magyar]{babel}

% AMS
\usepackage{amssymb,amsmath}

% Graphic packages
\usepackage{graphicx}

% Colors
\usepackage{color}
\usepackage[usenames,dvipsnames]{xcolor}

% Enumeration
\usepackage{enumitem}

% Links
\usepackage{hyperref}

\pagestyle{empty}

\begin{document}

\begin{center}
	{\Large B-spline tartományok}
\end{center}

\vskip 5mm

\section{Definíciók}

Tegyük fel, hogy a B-spline alapfüggvényeket az alábbi formában definiáljuk:

\begin{equation}\label{eq:base1}
  N_i^1(u) = \begin{cases}
	1, & \text{ha } u_i \leq u < u_{i+1}, \\
	0, & \text{egyébként},
\end{cases}
\end{equation}

\begin{equation}\label{eq:basek}
N_i^k(u) =
\dfrac{u - u_i}{u_{i+k-1} - u_i} N_i^{k-1}(u) +
\dfrac{u_{i+k} - u}{u_{i+k} - u_{i+1}} N_{i+1}^{k-1}(u).
\end{equation}

\noindent A B-spline görbe definíciója:

\begin{equation}\label{eq:curve}
\textbf{s}(u) = \sum_{i=0}^n N_i^k(u) \textbf{d}_i.
\end{equation}

\noindent Észrevételek:
\begin{itemize}
	\item A szakirodalomban a $k$ helyett gyakran $p$-t használnak, illetve $k$ és $(k - 1)$ helyett inkább $p$ és $(p + 1)$ szerepel.
	\item Az $i, k$ és $n$ tartományai szándékosan nem kerültek feltüntetésre, mert az a vizsgálat tárgya.
\end{itemize}

\section{Tartományok}

Egyszerűen belátható, ellenőrízhető (\ref{eq:base1} és \ref{eq:basek} alapján), hogy \(N_i^k(u) \Rightarrow u \in [u_i, u_{i + k}).\) (Az \(u_{i + k}\) így felhasználásra kerül, csak az $u$ nem veszi fel már, mint értéket.)

Tegyük fel, hogy adott $(n + 1)$ darab kontrollpontunk: \(\textbf{d}_0, \textbf{d}_1, \ldots \textbf{d}_n\). \Aref{eq:curve} és az előző megállapítás alapján $u \in [u_0, u_{n + k})$, mivel $0 \leq i \leq n$.

Vizsgáljuk meg, hogy $k$ milyen értékeket vehet fel! Ehhez tekintsük az alábbi eseteket!
\begin{itemize}
\item $k = 1$: \Aref{eq:base1} alapján a legkisebb felvehető érték, tehát már tudunk vele számolni. Mivel az alapfüggvény szakaszonként konstans, ezért nem görbét kapunk.
\item $k = 2$: Legkisebb felvehető értékként már használható. Egy kontrollpontot feltételezve ($n = 0$ esetén) az $\{u_0, u_1, u_2\}$ csomóértékek megadására van szükség.
\item $k = n$: A görbe számításához az $u \in [u_0, u_{n+k})$ megadásához szükséges csomóértékekre van szükségünk. A kontrollpontok indexei alapján ezek adottak.
\item $k = n + 1$: Az $N_n^{n+1}(u)$ eset vizsgálandó. Ehhez az $u_{n+k+1}$ csomóértékre is szükség lenne, amit viszont már nem tekintettünk adottnak.
\end{itemize}

Ezek alapján $(n + 1)$ kontrollpont esetén a $k$ értékét az $1 < k < n + 1$ tartományon adhatjuk meg.

\vskip 15mm

\noindent Miskolc, 2025. június 3.

\end{document}
