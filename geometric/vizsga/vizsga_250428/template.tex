\documentclass[a4paper]{article}

% Set margins
\usepackage[hmargin=2.5cm, vmargin=3cm]{geometry}

\frenchspacing

% Language packages
\usepackage[utf8]{inputenc}
\usepackage[T1]{fontenc}
\usepackage[magyar]{babel}

% AMS
\usepackage{amssymb,amsmath}

\usepackage{xcolor}

% Görbék 4
% Spline-ok 4
% Bézier görbe 4
% Felületek 4
% Képfeldolgozás 2
% Számítási feladat 6

\begin{document}

\pagestyle{empty}

Miskolci Egyetem, Matematikai Intézet

\hskip 10cm Név:

\medskip

\hskip 10cm Neptun-kód:

\begin{center}
   \large \textbf{Vizsga zárthelyi dolgozat\\
   GEOMETRIAI MODELLEZÉS (GEAGT232-M) c. tárgyból}
\end{center}

\bigskip

\noindent \textbf{1. Feladat} Kísérő triéder \textit{(4 pont)}
\begin{itemize}
	\item[(1)] Ismertesse az $\textbf{r}(t), t \in [a, b]$ térgörbe kísérőtriéderét!
	\item[(1)] Nevezze meg ennek vektorait, azok számítását,
	\item[(1)] síkjait.
	\item[(1)] Készítsen hozzá ábrát!
\end{itemize}

\bigskip

\noindent \textbf{2. Feladat}
Görbék paraméterezése \textit{(4 pont)}
\begin{itemize}
	\item[(4)] Mutassa be az interpolációs spline-ok paraméterezésének 4 tipikus esetét!
\end{itemize}

\bigskip

\noindent \textbf{3. Feladat}
Bézier görbe \textit{(4 pont)}
\begin{itemize}
	\item[(1)] Definiálja a Bernstein polinomot!
	\item[(1)] Írja fel a Bézier görbe paraméteres alakját Bernstein polinom segítségével!
	\item[(2)] Soroljon fel a Bézier görbe tulajdonságai közül legalább 4-et!
\end{itemize}

\bigskip

\noindent \textbf{4. Feladat}
Vonalfelületek \textit{(4 pont)}
\begin{itemize}
	\item[(1)] Írja le, hogy mit nevezünk vonalfelületeknek!
	\item[(2)] Adja meg a két tipikus származtatási/előállítási módjukat!
	\item[(1)] Készítsen ábrát!
\end{itemize}

\bigskip

\noindent \textbf{5. Feladat}
Hisztogram műveletek \textit{(2 pont)}
\begin{itemize}
	\item[(1)] Ismertesse a hisztogram széthúzás műveletét szürkeárnyalatos képek esetén!
	\item[(1)] Készítsen hozzá ábrát!
\end{itemize}

\bigskip

\noindent \textbf{6. Feladat}
Számítási feladat \textit{(6 pont)}

\medskip

Adottak a síkon a $(2, 1), (3, 3), (5, 2), (6, 4), (8, 4)$ pontok.
\begin{itemize}
	\item[(1)] Számítsa ki a Catmull-Rom spline pontjaihoz tartozó érintőket! 
	\item[(1)] Készítsen hozzá ábrát!
	\item[(1)] Számítsa ki a 3. ponthoz tartozó Bessel parabolát!
	\item[(1)] Számítsa ki a 3. pontban a Bessel érintőt!
	\item[(1)] Tekintse az első 4 pontot, mint kontroll pontot, és számítsa ki a Bézier görbe pontját $t = 0.4$ paraméternél (de Casteljau algoritmussal)!
	\item[(1)] Írja fel a Bézier görbe $t = 0.4$ paraméternél való kettévágásával adódó kontrollpontokat!
\end{itemize}

\bigskip

\noindent \textit{Ponthatárok: 0-11 elégtelen, 12-15 elégséges, 16-18 közepes, 19-21 jó, 22-24 jeles}

\newpage


Miskolci Egyetem, Matematikai Intézet

\hskip 10cm Név:

\medskip

\hskip 10cm Neptun-kód:

\begin{center}
	\large \textbf{Vizsga zárthelyi dolgozat\\
		GEOMETRIAI MODELLEZÉS (GEAGT232-M) c. tárgyból}
\end{center}

\bigskip

\noindent \textbf{1. Feladat} Görbék tulajdonságai \textit{(4 pont)}
\begin{itemize}
	\item[(1)] Sorolja fel a paraméteres térgörbék gyakran használt pontbeli tulajdonságait!
	\item[(1)] Mit nevezünk ívhossznak, és hogyan számítható?
	\item[(1)] Definiálja a görbületet!
	\item[(1)] Készítsen hozzá ábrát!
\end{itemize}

\bigskip

\noindent \textbf{2. Feladat}
Overhauser spline \textit{(4 pont)}
\begin{itemize}
	\item[(1)] Ismertesse az Overhauser spline-t!
	\item[(2)] Írja fel a paraméteres alakját!
	\item[(1)] Készítsen hozzá ábrát!
\end{itemize}

\bigskip

\noindent \textbf{3. Feladat}
Bézier görbe \textit{(4 pont)}
\begin{itemize}
	\item[(1)] Mutassa be a de Casteljau algoritmust!
	\item[(1)] Készítsen ábrát!
	\item[(2)] Soroljon fel a Bézier görbe tulajdonságai közül legalább 4-et!
\end{itemize}

\bigskip

\noindent \textbf{4. Feladat}
Mozgó görbe által súrolt felület \textit{(4 pont)}
\begin{itemize}
	\item[(1)] Hogyan írhatunk fel mozgó görbe által súrolt felületeket?
	\item[(1)] Írja fel a felület paraméteres alakját!
	\item[(1)] Készítsen ábrát!
	\item[(1)] Említsen néhány így előállítható felületet!
\end{itemize}

\bigskip

\noindent \textbf{5. Feladat}
Medián szűrő \textit{(2 pont)}
\begin{itemize}
	\item[(1)] Ismertesse a medián szűrő működését!
	\item[(1)] Készítsen hozzá ábrát!
\end{itemize}

\bigskip

\noindent \textbf{6. Feladat}
Számítási feladat \textit{(6 pont)}

\medskip

Adottak a síkon a $(2, 2), (3, 4), (6, 4), (8, 1)$ pontok.
\begin{itemize}
	\item[(1)] Ezeket kontrollpontoknak tekintve számítsa ki a Bézier görbe pontját a $t = 0.3$ paraméternél (de Casteljau algoritmussal)!
	\item[(1)] Készítsen hozzá ábrát!
	\item[(1)] Határozza meg a pontbeli érintőt!
	\item[(1)] Írja fel a pontbeli normálvektort!
	\item[(1)] Növelje a görbe fokszámát, és adja meg a hozzá tartozó kontrollpontokat!
	\item[(1)] A fokszámnöveléshez készítsen külön ábrát!
\end{itemize}

\bigskip

\noindent \textit{Ponthatárok: 0-11 elégtelen, 12-15 elégséges, 16-18 közepes, 19-21 jó, 22-24 jeles}

\end{document}
