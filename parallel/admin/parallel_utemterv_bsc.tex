\documentclass[a4paper,12pt]{article}

% Set margins
\usepackage[hmargin=2.5cm, vmargin=2cm]{geometry}

\frenchspacing

% Language packages
\usepackage[utf8]{inputenc}
\usepackage[T1]{fontenc}
\usepackage[magyar]{babel}

% AMS
\usepackage{amssymb,amsmath}

% Graphic packages
\usepackage{graphicx}

% Colors
\usepackage{color}
\usepackage[usenames,dvipsnames]{xcolor}

% Enumeration
\usepackage{enumitem}

% Links
\usepackage{hyperref}

\pagestyle{empty}

\begin{document}

\begin{center}
	{\Large ÜTEMTERV -- \texttt{GEMAK243-B}}

	\bigskip

	{\huge \textbf{Párhuzamos algoritmusok}}
	
	\medskip
	
	{\large Programtervező informatikus BSc,
	
	nappali tagozat, tavaszi félév}

	\medskip
	
	\textit{Előfeltétel: Objektum orientált programozás (GEIAL-313B)}
\end{center}

\vskip 5mm

\noindent \textbf{1. hét}: A párhuzamos algoritmusok alapvető fogalmai, tételei. PRAM, CREW, CRCW, EREW modellek áttekintése.
Számítási költség és hatékonyság jellemzése.
Brent tétele. Amdahl tétele.
Párhuzamos és konkurrens végrehajtás ábrázolási módjai.
Prefixszámítás.
Végrehajtás ütemezése.

\bigskip

\noindent \textbf{2. hét}: Párhuzamos végrehajtási módok. Hardveres és szoftveres környezet. \textit{Lost update} probléma. Szinkronizálás. Zárolási mechanizmusok. Kommunikáció. Memóriakezelés. Pipeline végrehajtás.

\bigskip

\noindent \textbf{3. hét}: Szálkezelés. POSIX szálak (\texttt{pthread} függvénykönyvtár). További szálkezelő függvénykönyvtárak áttekintése.

\bigskip

\noindent \textbf{4. hét}: A konkurrens és párhuzamos végrehajtás nyelvi szintű megjelenése. Az OpenMP függvénykönyvtár.

\bigskip

\noindent \textbf{5. hét}: Szekvenciális programok kommunikációja (\textit{CSP, Communicating Sequential Processes}). Az aktor modell. Az MPI szabvány. Az Open MPI áttekintése, működésének és használatának bemutatása.

\bigskip

\noindent \textbf{6. hét}: Rendező algoritmusok: Összefésülés. Batcher-féle páros-páratlan összefésülés. Összefésülő rendezés. Gyorsrendezés. Buborék és Shell rendezés. Leszámláló rendezés.

\bigskip

\noindent \textbf{7. hét}: Numerikus módszerek: mátrix szorzás, mátrix inverz számítás, numerikus integrálás.

\bigskip

\noindent \textbf{8. hét}: Statisztika: leíró statisztikák számítása, hisztogram számítása, Monte-Carlo módszer.

\bigskip

\noindent \textbf{9. hét}: Lekérdezések: Szövegláncok, szövegek keresése. Halmazok, halmazműveletek.

\bigskip

\noindent \textbf{10. hét}: Gráfalgoritmusok: összefüggő komponensek keresése. Minimális feszítőfa számítása. Tranzitív lezárt. Legrövidebb utak számítása.

\bigskip

\noindent \textbf{11. hét}: Grafikai algoritmusok: konvex burok meghatározása. Kitöltő algoritmusok. Konvolúciós szűrők megvalósítása.

\bigskip

\noindent \textbf{12. hét}: Optimalizálás: Genetikus algoritmusok, Részecskeraj optimalizáció (\textit{PSO, Particle Swarm optimization}).

\bigskip

\noindent \textbf{13. hét}: Klaszterezés, osztályozás: K-közép algoritmus (\textit{K-means}). Hierarchikus agglomeratív klaszterezés.

\bigskip

\noindent \textbf{14. hét}: Klasszikus mesterséges intelligencia algoritmusok: Minimax algoritmus, Alfa-béta metszés.

\newpage

\noindent \textbf{Az aláírás megszerzésének feltétele}

\noindent A gyakorlatokon való aktív részvétel. Egyéni feladat elkészítése a \texttt{pthread}, az OpenMPI függvénykönyvtárak és egy szabadon választott (párhuzamos programok készítéséhez használt) technológia segítségével. Az elvégzett mérések és a kapott eredmények dokumentálása és bemutatása.

\vskip 1cm

\noindent \textbf{A vizsga}

\noindent A vizsga írásbeli lesz, mely egyaránt tartalmaz elméleti és gyakorlati kérdéseket is.

\vskip 1cm

\noindent \textbf{Ponthatárok}

\noindent A dolgozatokon maximálisan 12 pontot lehet szerezni. Az érdemjegyekre az alábbi ponthatárok vonatkoznak.

\begin{center}
\begin{tabular}{|c|c|}
pont & érdemjegy \\
\hline
0-5 & 1 \\
6 & 2 \\
7-8 & 3 \\
9-10 & 4 \\
11-12 & 5 \\
\hline
\end{tabular}
\end{center}

\vskip 18mm

\hskip 10cm Piller Imre

\hskip 7cm Alkalmazott Matematikai Intézeti Tanszék

\vskip 5mm

\noindent Miskolc, 2022. január 28.

\end{document}
