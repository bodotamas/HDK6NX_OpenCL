\documentclass[a4paper,12pt]{article}

% Set margins
\usepackage[hmargin=2.5cm, vmargin=3cm]{geometry}

\frenchspacing

% Language packages
\usepackage[utf8]{inputenc}
\usepackage[T1]{fontenc}
\usepackage[magyar]{babel}

% AMS
\usepackage{amssymb,amsmath}

% Graphic packages
\usepackage{graphicx}

% Colors
\usepackage{color}
\usepackage[usenames,dvipsnames]{xcolor}

% Enumeration
\usepackage{enumitem}

% Links
\usepackage{hyperref}

\pagestyle{empty}

\begin{document}

\begin{center}
	{\Large ÜTEMTERV -- \texttt{GEMAK132ML}}

	\bigskip

	{\huge \textbf{Párhuzamos algoritmusok}}
	
	\medskip
	
	{\large Mérnökinformatikus MSc, levelező tagozat, őszi félév}
\end{center}

\bigskip

\noindent \textbf{1. hét}

\noindent PRAM, CREW, CRCW, EREW modellek áttekintése.
Számítási költség és hatékonyság jellemzése.
Brent tétele. Amdahl tétele.
Párhuzamos és konkurrens végrehajtás ábrázolási módjai.
Végrehajtás ütemezése. Memóriaszervezés, globális- és lokális memória elérése.
Hálózati topológiák. Üzenetátadás.
Példák párhuzamos végrehajtható problémákra, azok megoldási algoritmusaira.

\bigskip

\noindent \textbf{2. hét}

\noindent Processzek és szálak az operációs rendszerekben.
POSIX folyamatmodell.
A \texttt{pthread} függvénykönyvtár áttekintése példákkal.

\bigskip

\noindent \textbf{3. hét}

\noindent Az OpenMP függvénykönyvtár elemeinek bemutatása.
\texttt{for} ciklusok párhuzamos végrehajtása.
Az osztott memória elérése.
A párhuzamosítás hatékonyságának vizsgálata.
Az OpenMPI függvénykönyvtár áttekintése.
Algoritmizálási problémák megoldása a függvénykönyvtár segítségével.
Futási idők mérése.
Kapott eredmények ábrázolása, összehasonlítása.

\bigskip

\noindent \textbf{4. hét}

\noindent A \texttt{pthread}, OpenMP és OpenMPI összevetése. További példák vizsgálata párhuzamos algoritmusokra. Egyéni feladatok bemutatása.

\vskip 1cm

\noindent \textbf{Az aláírás megszerzésének feltétele}

\noindent Egy előre egyeztetett számítási probléma megoldása párhuzamos algoritmus segítségével, C nyelven implementálva \texttt{pthread}, OpenMP és OpenMPI függvénykönyvtár használatával. A probléma, az algoritmus, a három implementáció és a hozzájuk kapcsolódó mérések bemutatása.

\vskip 1cm

\noindent \textbf{A vizsga}

\noindent A vizsga írásbeli lesz, mely egyaránt tartalmaz elméleti és gyakorlati kérdéseket is.

\vskip 18mm

\hskip 10cm Piller Imre

\hskip 7cm Alkalmazott Matematikai Intézeti Tanszék

\vskip 5mm

\noindent Miskolc, 2021. augusztus 27.

\end{document}
