\documentclass[a4paper]{article}

% Set margins
\usepackage[hmargin=2.5cm, vmargin=3cm]{geometry}

\frenchspacing

% Language packages
\usepackage[utf8]{inputenc}
\usepackage[T1]{fontenc}
\usepackage[magyar]{babel}

% AMS
\usepackage{amssymb,amsmath}

\usepackage{xcolor}

\begin{document}

\pagestyle{empty}

Miskolci Egyetem, Matematikai Intézet

\hskip 10cm Név:

\medskip

\hskip 10cm Neptun-kód:

\begin{center}
   \large \textbf{Vizsga zárthelyi dolgozat - \texttt{240206-0000} \\
   SZÁMÍTÓGÉPI GRAFIKA (GEAGT131-B) c. tárgyból}
\end{center}

\bigskip

\noindent \textbf{1. Feladat}
Ábrázolja az RGB színkockát, és jelölje a csúcspontokhoz tartozó színeket!


\bigskip

\noindent \textbf{2. Feladat}
Ismertesse a Bresenham-féle szakaszrajzoló algoritmus működését!

\bigskip

\noindent \textbf{3. Feladat}
Ismertesse a nézeti gúla fogalmát, és a hozzá tartozó paramétereket!


\bigskip

\noindent \textbf{4. Feladat}
Részletezze a textúrázásnál jelentkező mintavételezési problémákat, és azok lehetséges megoldásait!


\bigskip

\noindent \textbf{5. Feladat}
A síkbeli $(5, -3)$ pontot 15 fok/másodperc sebességgel szeretnénk forgatni az origó körül. A $t = 0$ időpontból indulunk. Hol lesz a pont a $t = 5$, $t = 6$ és $t = 7$ másodperchez tartozó időpontokban?

\bigskip

\noindent \textbf{6. Feladat}
Adott a 3 dimenziós térben egy háromszög az $(1, 0, -2), (3, 2, 4)$ és $(5, 5, 5)$ csúcspontokkal. Számítsa ki a háromszög normál vektorát!

\bigskip

\noindent \textbf{7. Feladat}
Adott a 3 dimenziós térben az $(5, 9, 7)$ pont. Perspektivikus vetítést feltételezve hova kerül a vetített pont, hogy ha a képsík a $z = 0$ sík, és a nézőpont a képsík mögött 2 egységgel helyezkedik el?

\bigskip

\noindent \textbf{8. Feladat}
Írja fel annak a transzformációnak a transzformációs mátrixát, amely a 3 dimenziós tér pontjait az $x = y$ síkra vetíti!

\bigskip

\bigskip

\begin{itemize}
	\item Az ismertetéseket és az elvégzett számításokat ábrákkal illusztrálva részletezze!
	\item az 1-4. feladatok 1 pontosak, az 5-8. feladatok 2 pontosak.
\end{itemize}

\bigskip

\noindent \textit{Ponthatárok: 0-5 elégtelen, 6 elégséges, 7-8 közepes, 9-10 jó, 11-12 jeles}

\end{document}

