\documentclass[a4paper,12pt]{article}

% Set margins
\usepackage[hmargin=2.5cm, vmargin=2cm]{geometry}

\frenchspacing

% Language packages
\usepackage[utf8]{inputenc}
\usepackage[T1]{fontenc}
\usepackage[magyar]{babel}

% AMS
\usepackage{amssymb,amsmath}

% Graphic packages
\usepackage{graphicx}

% Colors
\usepackage{color}
\usepackage[usenames,dvipsnames]{xcolor}

% Enumeration
\usepackage{enumitem}

% Links
\usepackage{hyperref}

\pagestyle{empty}

\begin{document}

\begin{center}
	{\Large ÜTEMTERV -- \texttt{GEMAK259-B}}

	\bigskip

	{\huge \textbf{Párhuzamos eszközök programozása}}
	
	\medskip
	
	{\large Programtervező informatikus BSc,
	
	nappali tagozat, tavaszi félév}

	\medskip
	
	\textit{Előfeltétel: Párhuzamos algoritmusok (GEMAK243-B)}
\end{center}

\vskip 5mm

\noindent \textbf{1. hét}: A párhuzamos végrehajtásra alkalmas eszközök változatai, fejlődésük. Alapvető fogalmak. Az OpenCL nyelv áttekintése. A platform és az eszközök adatainak lekérdezése. Példa vektorok összeadására.

\bigskip

\noindent \textbf{2. hét}: Memóriakezelés. Adattípusok. Eszközök hibakereséshez, profilozáshoz. Mátrixszorzás implementálása.

\bigskip

\noindent \textbf{3. hét}: Szinkronizációs módszerek. Termelő-fogyasztó probléma. Adatfolyam feldolgozás. Cella alapú számítások.

\bigskip

\noindent \textbf{4. hét}: A teljes OpenCL API áttekintése. Az OpenCL használata különböző programozási nyelvekkel (például C++, Python).

\bigskip

\noindent \textbf{5. hét}: Rendezőalgoritmusok. Páros-páratlan rendezés. Összefésülő rendezés alternatív megvalósításai.

\bigskip

\noindent \textbf{6. hét}: Kriptográfiai algoritmusok. Hash számítás, visszafejtés.

\bigskip

\noindent \textbf{7. hét}: Tömörítési módszerek párhuzamosítási lehetőségei. Huffman kódolás implementálása.

\bigskip

\noindent \textbf{8. hét}: Gépi tanulási módszerek. Mesterséges neurális hálózatok párhuzamos betanítása.

\bigskip

\noindent \textbf{9. hét}: Egy dimenziós jelfeldolgozás (például hang). Fourier transzformáció és inverze (folytonos és diszkrét esetben).

\bigskip

\noindent \textbf{10. hét}: Képformátumok kezelése. Az OpenCL és az OpenGL összekapcsolási módjai. Sugárkövető algoritmusok készítése.

\bigskip

\noindent \textbf{11. hét}: Képfeldolgozási problémák párhuzamosítási lehetőségei. Transzformációk (például forgatás). Hisztogramok számítása. Konvolúciós szűrők implementálása (például átlagoló-, Gauss-, medián szűrők). Az OpenCV függvénykönyvtár áttekintése.

\bigskip

\noindent \textbf{12. hét}: Fizikai szimulációk készítése. Részecskerendszerek. Cella alapú módszerek.

\bigskip

\noindent \textbf{13. hét}: Alkalmazások, függvénykönyvtárak hatékonyabbá tétele OpenCL segítségével. Nagyobb, nyílt forráskódú OpenCL-t használó alkalmazások áttekintése.

\bigskip

\noindent \textbf{14. hét}: Kitekintés. Tudományos kutatások és eredményeik a témakörben.

\newpage

\noindent \textbf{A tárgy teljesítésének módja}

\medskip

\noindent Gyakorlatokon való aktív részvétel. Egy saját témában választott feladat elkészítése a tárgyban bemutatott eszközkészlettel. A hozzá tartozó specifikáció, programkód, mérések, dokumentáció bemutatása. A tárgy az ezek alapján kapott gyakorlati jeggyel zárul.

\vskip 18mm

\hskip 10cm Piller Imre

\hskip 7cm Alkalmazott Matematikai Intézeti Tanszék

\vskip 5mm

\noindent Miskolc, 2023. február 23.

\end{document}
